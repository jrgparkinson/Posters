\documentclass[]{article}

\begin{document}

\noindent \textbf{Brine rejection due to gravity drainage through porous mushy sea ice}

\noindent Sea ice is a porous material composed of ice crystals and interstitial brine; a mushy layer. The dense brine tends to sink through the ice, driving convection. Downwelling at the edge of convective cells leads to the development of narrow, entirely liquid channels, through which cold saline brine is efficiently rejected into the underlying ocean. This brine rejection provides an important buoyancy forcing on the ocean, and can have important consequences
for the internal structure and properties of sea ice.

We consider 2-D numerical simulations of ice formation and convective brine rejection in a narrow Hele-Shaw cell.Adaptive Mesh Refinement, implemented via the Chombo framework, allows us to resolve narrow brine channels whilst integrating over many months of ice growth. The convective desalination of sea ice promotes increased internal solidification, and we find that convective brine drainage is restricted to a narrow porous layer at the ice-ocean interface, which evolves as the ice layer grows thicker over time. Away from this interface, stagnant sea ice consists of a network of previously active brine channels which retain higher solute concentrations than the surrounding ice. We investigate the response of these remnant brine channels to changes in atmospheric and oceanic conditions. Surface warming can increase the porosity in the upper layers of the ice, allowing blocked brine channels to drain into the ocean. Meanwhile, basal melt due to ocean warming can lead to the formation of a buoyant layer of fresh melt water which inhibits the transport of salt from the ice to the underlying ocean. We consider the potential implications for nutrient transport, sea ice ecology, and parametrizations of ice-ocean brine fluxes. 
\end{document}